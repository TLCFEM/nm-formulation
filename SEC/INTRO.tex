\section{Introduction}
Beam-column elements are commonly used to simulate the dynamic response of large-scale structures, such as multi-storey buildings and bridges, because of their high computational efficiency compared to continuum finite elements.
By allowing inelastic deformations to distribute within the span, these elements could simulate the spread of inelastic deformations within beams and columns with reasonable accuracy.
This leads to development of distributed inelasticity beam elements, with displacement-based \citep[e.g.,][]{Bathe:96:FEP,Crisfield:97:Book2,Zienkiewicz:Taylor:05:FEM}, force-based \citep[e.g.,][]{Spacone:et:al:96:Fibre,Alemdar2005,Addessi2007,Sideris2016} and mixed formulations \citep[e.g.,][]{Neuenhofer:Filippou:97:Evaluation,Petrangeli:Ciampi:97:Equilibrium,Hjelmstad:Taciroglu:02:Mixed,Taylor:et:al:03:MixedForm,Nukala:White:04:Mixed,DallAsta:Zona:04:Mixed,Lee:Filippou:09:SecInversion}.

In the absence of distributed loads within its body, the element, particularly when used as a column, would likely have its inelastic deformations occur at its ends, where bending moments are the largest.
The efficiency of distributed inelasticity can, therefore, be further optimised by monitoring inelastic deformations only at element ends \citep[e.g.,][]{Attalla:et:al:84:SpreadOP,Scott:Fenves:06:BeamWithHinges,Lee:Filippou:09:SIZE,Scott:Ryan:13:PlasticHinge}.
However, these elements rely on the concept of plastic hinge length, which could be difficult to be parametrised against state variables at element ends, especially for reinforced concrete frames \citep{Park1975,Scott1996}.
This difficulty can be removed by using the concept of concentrated (or lumped) plasticity, where plastic hinges are modelled with zero length.
Such elements are typically called \emph{concentrated plasticity elements}.

The earliest concentrated plasticity elements with this concept include the one-component model \citep{Giberson:67:OneComponent}.
This element was later extended to incorporate the $N$-$M$ interaction (yield surface) via resultant plasticity theory \citep{Orbison:et:al:82:YieldSurface}.
In terms of computing the $N$-$M$ interaction surface, an iterative design procedure was presented by \citet{Liu:et:al:12:CrossSection} for various types of sections.
To incorporate strain hardening, the concept of multiple yield surfaces such as loading and bounding surfaces \citep[e.g.,][]{Iwan:67:Yielding,Mroz:67:AnisoHardening,Mroz:69:GeneralHardening,Dafalias:Popov:75:YieldSurface,Dafalias:Popov:77:YieldSurface} in classic plasticity theory was also adopted \citep[e.g.,][]{Hilmy:Abel:85:PHElement,Hajjar:Gourley:97:PHElement}, usually for modelling steel--concrete composite members.
Among those, the beam model by \citet{Powell:Chen:86:GeneralizedPH} adopts discrete plastic hinges attaching to both end nodes, such that plasticity on two plastic hinges can evolve independently. Thus, each plastic hinge acts as a zero-length connector connecting `internal node' to `external node'.
By borrowing the concept of the generalised plasticity theory \citep{Auricchio1994}, \citet{Kostic:et:al:11:EfficientBCElem0,Kostic:et:al:13:EfficientBCElem,Kostic:et:al:16:GenPlasticity} further extended the formulation by removing the need of internal nodes and proposed a two-node element. Return map algorithms are typically used to ensure element forces stay on the yield surface upon yielding.

Combing the $N$-$M$ interaction and the generalised plasticity theory in the concentrated plasticity elements would cause some challenges given that the $N$-$M$ interaction at each beam end is independent of the other end, but both ends are linked to each other as they share an identical axial force.
In the existing formulation \citep{Kostic:et:al:11:EfficientBCElem0,Kostic:et:al:13:EfficientBCElem,Kostic:et:al:16:GenPlasticity}, due to the adoption of two independent sets of plasticity variables --- one for each node, numerical instability and bifurcation might result when both ends experience the same axial plasticity history. A careful examination reveals that local Jacobian would become singular under certain circumstances.
Due to the coupling between two ends, the hardening rules defined in a conventional manner cannot recover the desired hardening behaviour.
Furthermore, since the hardening parameters do not possess any physical implication, the calibration for their values for a desirable isotropic and kinematic hardening behaviour at either element end is also error--prone.

In this work, we revisit the generalised plasticity framework of the concentrated plasticity beam element and reformulate the $N$-$M$ interactions at each element end such that a desired isotropic and kinematic hardening behaviour can be achieved.
A new definition for axial plastic deformation is also introduced to address the bifurcation issue in the previous formulation.
The reformulation provides a solid theoretical framework that can be extended by adopting more complex plasticity rules to simulate frame members made of various materials.

In the following, the plasticity framework is first introduced with a return mapping type local integration algorithm.
A comparison between the previous \citep{Kostic:et:al:11:EfficientBCElem0} and revised models are then presented with discussions of existing issues and the corresponding improvements.
Numerical examples are investigated to validate and showcase the capability and performance of the proposed formulation.
%
%The conventional plasticity theory is often applied to strain and stress at each integration point. Similar concepts can be applied to other quantities such as cross--sectional/elemental deformation and resultant once the three base stones of plasticity theory, namely, yield function, flow rule and hardening law, are well defined. Based on this idea, \citet{Lubliner1993} proposed the generalised plasticity theory and \citet{Auricchio1994} applied it to membrane/plate problems. Such a framework provides great flexibility in terms of defining nonlinear plastic behaviour of elements and improves numerical performance significantly as there are no integration points involved, the state determination is essentially a single step plasticity integration.
%
%A two--node beam element developed based on the generalised plasticity model is an appealing tool in seismic/structural engineering. It minimises computational cost as there are no sections or material integration points involved. The efficiency of such types of beam elements is significantly higher than traditional displacement/force based fibre type elements. It provides an easy approach to simulate zero-length end plastic hinges. However, combing the $N$-$M$ interaction and the generalised plasticity theory in beam elements introduces new challenges given that the $N$-$M$ interaction at each beam end is relatively independent from the other end, but both ends are linked with each other as they share the \textbf{identical} axial force.
%
%A similar element has been proposed previously \citep{Kostic2013,Kostic2016}. However, its formulation appears to be problematic. In this work, we revisit the generalised plasticity framework and reformulate the previously proposed two--node beam element to enable $N$-$M$ interactions at each beam end and recover the desired both isotropic and kinematic hardening behaviour. By correct the physical implication of axial plastic deformation, a new definition is introduced to remove numerical instability issues the previous formulation has when both ends experience the identical plasticity history.
%
%The plasticity framework is first introduced with a return mapping type local integration algorithm. A comparison between the previous and revised models are then presented with discussions of existing issues and the corresponding improvements. Numerical examples are investigated to validate and showcase the capability and performance of the proposed element.