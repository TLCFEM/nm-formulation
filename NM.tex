\documentclass[3p,sort&compress,11pt,fleqn,review]{elsarticle}
\bibliographystyle{elsarticle-num-names}
\usepackage{amsmath,amsfonts,amssymb,siunitx,float,subcaption,setspace,booktabs,diagbox,bm,tikz,mathtools,tikz-3dplot,pgfplots,url}
\usepackage{algorithm}
\usepackage{algpseudocode}
\usepackage{lineno}
\usetikzlibrary{shapes.geometric}
\pgfplotsset{compat=1.8}
\journal{}
\newcommand*{\mb}[1]{\bm{#1}}
\newcommand*{\mT}{\mathrm{T}}
\newcommand*{\md}[1]{\mathrm{d}#1}
\newcommand*{\eqsref}[1]{Eq.~(\ref{#1})}
\newcommand*{\figref}[1]{Fig.~\ref{#1}}
\newcommand*{\tabref}[1]{Table~\ref{#1}}
\newcommand*{\algoref}[1]{Algorithm~\ref{#1}}
\newcommand*{\secref}[1]{\S~\ref{#1}}
\newcommand*{\diag}[1]{\text{diag}\left(#1\right)}
\newcommand*{\sign}[1]{\text{sign}\left(#1\right)}
\newcommand*{\dev}[1]{\text{dev}\left(#1\right)}
\newcommand*{\tr}[1]{\text{trace}\left(#1\right)}
\newcommand*{\ddfrac}[2]{\dfrac{\md{#1}}{\md{#2}}}
\newcommand*{\pdfrac}[2]{\dfrac{\partial{#1}}{\partial{#2}}}
\newcommand{\bsigma}{\mb{\sigma}}
\newcommand{\bvarepsilon}{\mb{\varepsilon}}
\newcommand{\beeta}{\mb{\eta}}
\newcommand{\bn}{\mb{n}}
\newcommand{\balpha}{\mb{\alpha}}
\newcommand{\bbeta}{\mb{\beta}}
\newcommand{\bgamma}{\mb{\gamma}}
\newcommand{\bq}{\mb{q}}
\newcommand{\bs}{\mb{s}}
\newcommand{\bc}{\mb{c}}
\newcommand{\be}{\mb{e}}
\newcommand{\bv}{\mb{v}}
\DeclarePairedDelimiter\abs{\lvert}{\rvert}
\DeclarePairedDelimiter\norm{\lVert}{\rVert}
\makeatletter
\let\oldabs\abs
\def\abs{\@ifstar{\oldabs}{\oldabs*}}
\let\oldnorm\norm
\def\norm{\@ifstar{\oldnorm}{\oldnorm*}}
\newenvironment{breakablealgorithm}
{\begin{center}
\refstepcounter{algorithm}
\hrule height.8pt depth0pt \kern2pt
\renewcommand{\caption}[2][\relax]{
{\raggedright\textbf{\ALG@name~\thealgorithm} ##2\par}
\ifx\relax##1\relax
\addcontentsline{loa}{algorithm}{\protect\numberline{\thealgorithm}##2}
\else
\addcontentsline{loa}{algorithm}{\protect\numberline{\thealgorithm}##1}
\fi
\kern2pt\hrule\kern2pt
}}{\kern2pt\hrule\relax
\end{center}}
\begin{document}
\linenumbers
\begin{abstract}
\begin{linenumbers}
The concept of generalised plasticity theory provides an appealing approach to developing efficient finite element models which could improve computational performance and allow flexibility in customising plastic response.
In this work, we revisit its application in developing frame elements and reformulate a concentrated plasticity frame element to allow its axial force to interact with its end moments ($N$-$M$ interaction).
The proposed element formulation relies on the physical implication of axial plasticity and updates its definition accordingly.
By which, a single axial plasticity history can participate in plasticity evolution and $N$-$M$ interaction at both ends.
The new formulation also redefines the hardening rules to recover desired hardening behaviour, which is not available in the previous formulation.
In terms of numerical performance, the new formulation removes the potential bifurcation issue when both ends experience identical pure axial plasticity history.
The proposed frame element is a high--performing alternative to simulate zero--length plastic hinge frame members that are frequently used in simulations in seismic engineering.
\end{linenumbers}
\end{abstract}
\begin{keyword}
concentrated plasticity frame element\sep
$N$-$M$ interaction\sep
generalised plasticity theory\sep
hysteresis model
\end{keyword}
\begin{frontmatter}
\title{Reformulation of Concentrated Plasticity Frame Element With $N$-$M$ Interaction and Generalised Plasticity}
\author[add1]{Theodore~L.~Chang\corref{tlc}}\ead{tlcfem@gmail.com}
\author[add2]{Chin-Long~Lee}
\cortext[tlc]{corresponding author}
\address[add1]{IRIS Adlershof, Humboldt-Universität zu Berlin, Berlin, Germany, 12489.}
\address[add2]{Department of Civil and Natural Resources Engineering, University of Canterbury, Christchurch, New Zealand, 8041.}
\end{frontmatter}
\section{Introduction}
Beam-column elements are commonly used to simulate the dynamic response of large-scale structures, such as multi-storey buildings and bridges, because of their high computational efficiency compared to continuum finite elements.
By allowing inelastic deformations to distribute within the span, these elements could simulate the spread of inelastic deformations within beams and columns with reasonable accuracy.
This leads to development of distributed inelasticity beam elements, with displacement-based \citep[e.g.,][]{Bathe:96:FEP,Crisfield:97:Book2,Zienkiewicz:Taylor:05:FEM}, force-based \citep[e.g.,][]{Spacone:et:al:96:Fibre,Alemdar2005,Addessi2007,Sideris2016} and mixed formulations \citep[e.g.,][]{Neuenhofer:Filippou:97:Evaluation,Petrangeli:Ciampi:97:Equilibrium,Hjelmstad:Taciroglu:02:Mixed,Taylor:et:al:03:MixedForm,Nukala:White:04:Mixed,DallAsta:Zona:04:Mixed,Lee:Filippou:09:SecInversion}.

In the absence of distributed loads within its body, the element, particularly when used as a column, would likely have its inelastic deformations occur at its ends, where bending moments are the largest.
The efficiency of distributed inelasticity can, therefore, be further optimised by monitoring inelastic deformations only at element ends \citep[e.g.,][]{Attalla:et:al:84:SpreadOP,Scott:Fenves:06:BeamWithHinges,Lee:Filippou:09:SIZE,Scott:Ryan:13:PlasticHinge}.
However, these elements rely on the concept of plastic hinge length, which could be difficult to be parametrised against state variables at element ends, especially for reinforced concrete frames \citep{Park1975,Scott1996}.
This difficulty can be removed by using the concept of concentrated (or lumped) plasticity, where plastic hinges are modelled with zero length.
Such elements are typically called \emph{concentrated plasticity elements}.

The earliest concentrated plasticity elements with this concept include the one-component model \citep{Giberson:67:OneComponent}.
This element was later extended to incorporate the $N$-$M$ interaction (yield surface) via resultant plasticity theory \citep{Orbison:et:al:82:YieldSurface}.
In terms of computing the $N$-$M$ interaction surface, an iterative design procedure was presented by \citet{Liu:et:al:12:CrossSection} for various types of sections.
To incorporate strain hardening, the concept of multiple yield surfaces such as loading and bounding surfaces \citep[e.g.,][]{Iwan:67:Yielding,Mroz:67:AnisoHardening,Mroz:69:GeneralHardening,Dafalias:Popov:75:YieldSurface,Dafalias:Popov:77:YieldSurface} in classic plasticity theory was also adopted \citep[e.g.,][]{Hilmy:Abel:85:PHElement,Hajjar:Gourley:97:PHElement}, usually for modelling steel--concrete composite members.
Among those, the beam model by \citet{Powell:Chen:86:GeneralizedPH} adopts discrete plastic hinges attaching to both end nodes, such that plasticity on two plastic hinges can evolve independently. Thus, each plastic hinge acts as a zero-length connector connecting `internal node' to `external node'.
By borrowing the concept of the generalised plasticity theory \citep{Auricchio1994}, \citet{Kostic:et:al:11:EfficientBCElem0,Kostic:et:al:13:EfficientBCElem,Kostic:et:al:16:GenPlasticity} further extended the formulation by removing the need of internal nodes and proposed a two-node element. Return map algorithms are typically used to ensure element forces stay on the yield surface upon yielding.

Combing the $N$-$M$ interaction and the generalised plasticity theory in the concentrated plasticity elements would cause some challenges given that the $N$-$M$ interaction at each beam end is independent of the other end, but both ends are linked to each other as they share an identical axial force.
In the existing formulation \citep{Kostic:et:al:11:EfficientBCElem0,Kostic:et:al:13:EfficientBCElem,Kostic:et:al:16:GenPlasticity}, due to the adoption of two independent sets of plasticity variables --- one for each node, numerical instability and bifurcation might result when both ends experience the same axial plasticity history. A careful examination reveals that local Jacobian would become singular under certain circumstances.
Due to the coupling between two ends, the hardening rules defined in a conventional manner cannot recover the desired hardening behaviour.
Furthermore, since the hardening parameters do not possess any physical implication, the calibration for their values for a desirable isotropic and kinematic hardening behaviour at either element end is also error--prone.

In this work, we revisit the generalised plasticity framework of the concentrated plasticity beam element and reformulate the $N$-$M$ interactions at each element end such that a desired isotropic and kinematic hardening behaviour can be achieved.
A new definition for axial plastic deformation is also introduced to address the bifurcation issue in the previous formulation.
The reformulation provides a solid theoretical framework that can be extended by adopting more complex plasticity rules to simulate frame members made of various materials.

In the following, the plasticity framework is first introduced with a return mapping type local integration algorithm.
A comparison between the previous \citep{Kostic:et:al:11:EfficientBCElem0} and revised models are then presented with discussions of existing issues and the corresponding improvements.
Numerical examples are investigated to validate and showcase the capability and performance of the proposed formulation.
%
%The conventional plasticity theory is often applied to strain and stress at each integration point. Similar concepts can be applied to other quantities such as cross--sectional/elemental deformation and resultant once the three base stones of plasticity theory, namely, yield function, flow rule and hardening law, are well defined. Based on this idea, \citet{Lubliner1993} proposed the generalised plasticity theory and \citet{Auricchio1994} applied it to membrane/plate problems. Such a framework provides great flexibility in terms of defining nonlinear plastic behaviour of elements and improves numerical performance significantly as there are no integration points involved, the state determination is essentially a single step plasticity integration.
%
%A two--node beam element developed based on the generalised plasticity model is an appealing tool in seismic/structural engineering. It minimises computational cost as there are no sections or material integration points involved. The efficiency of such types of beam elements is significantly higher than traditional displacement/force based fibre type elements. It provides an easy approach to simulate zero-length end plastic hinges. However, combing the $N$-$M$ interaction and the generalised plasticity theory in beam elements introduces new challenges given that the $N$-$M$ interaction at each beam end is relatively independent from the other end, but both ends are linked with each other as they share the \textbf{identical} axial force.
%
%A similar element has been proposed previously \citep{Kostic2013,Kostic2016}. However, its formulation appears to be problematic. In this work, we revisit the generalised plasticity framework and reformulate the previously proposed two--node beam element to enable $N$-$M$ interactions at each beam end and recover the desired both isotropic and kinematic hardening behaviour. By correct the physical implication of axial plastic deformation, a new definition is introduced to remove numerical instability issues the previous formulation has when both ends experience the identical plasticity history.
%
%The plasticity framework is first introduced with a return mapping type local integration algorithm. A comparison between the previous and revised models are then presented with discussions of existing issues and the corresponding improvements. Numerical examples are investigated to validate and showcase the capability and performance of the proposed element.
\input{SEC/THEORY}
\section{Numerical Examples}
\subsection{Validation}
As an improved reformulation, in this section, we mainly demonstrate the fact that the proposed formulation is able to produce a desired hardening response.
The cantilever beam has a unit length $L=1$, and all rigidities and yield forces are set to unity, that is, $EA=1$, $EI_z=1$, $P^y=1$ and $M^y_z=1$.
Units are dropped for brevity.
\subsubsection{Isotropic Hardening}
The interaction surface is defined as
\begin{gather}
\Phi=\left(\dfrac{\overline{P}-\overline{\beta}_P}{1+0.1\overline{\alpha}}\right)^2+2\left(\dfrac{\overline{M}_z-\overline{\beta}_{M_z}}{1+0.1\overline{\alpha}}\right)^2-1,
\end{gather}
implying $H=0.1$.
The theoretical hardening ratio can be computed according to \eqsref{eq:nm_eqv_iso_hardening}.
\figref{fig:iso_hardening_a}, \figref{fig:iso_hardening_b} and \figref{fig:iso_hardening_c} present uniaxial cyclic isotropic hardening response under axial force, shear force, and end moment, respectively.
Due to numerical discretisation, some points fall in the transition region(s) and lead to intermediate stiffness values.
They can be ignored.
The desired linear isotropic hardening ratio is properly generated no matter which end yields.
As aforementioned, it is possible to define different hardening functions $h\left(\overline{\alpha}\right)$ for different components to achieve an identical hardening ratio for single DoF loading cases.
However, a unified hardening function ensures the interaction surface grows in size uniformly (isotropic scaling).
\subsubsection{Kinematic Hardening}
The adopted linear kinematic hardening rule is defined as
\begin{gather}
\dot{\overline{\bbeta}}=K\dot{\overline{\be}^p}=0.1\dot{\overline{\be}^p}.
\end{gather}
The theoretical hardening ratio can be computed according to \eqsref{eq:nm_eqv_kin_hardening}.
\figref{fig:kin_hardening_a}, \figref{fig:kin_hardening_b} and \figref{fig:kin_hardening_c} present uniaxial cyclic kinematic hardening response under axial force, shear force, and end moment, respectively.
The target linear kinematic hardening behaviour is also properly generated.
The hardening ratio stays constant (as assigned) for all DoFs, no matter whether one end yields or both ends yield.
\subsection{Calibration}
As mentioned before, in \eqsref{nm:kinematics}, the elemental deformation essentially consists of end-section curvatures and axial strain. It is possible to calibrate the hardening behaviour according to section analysis results in absence of the corresponding experimental data.

An example is presented to show the process by applying moment to a rectangular section.
The section has a geometry of $w\times{}b=12\times1$, the theoretical shape factor of which is \num{1.5}.
This leads to the ultimate moment $M^{ult}=1.5M^y$ assuming material response is purely elastic and perfectly plastic.
By choosing material yield stress $\sigma^y=1$, elastic modulus $E=1$, one can compute $EI_z=2$ and $M^y=2$.

For numerical convenience, \SI{1}{\percent} isotropic hardening is issued to both material models used in section analysis and the proposed beam element.
To reproduce the Bauschinger effect, kinematic hardening is activated with the following parameter: $K_a=1.8$ and $K_b=0.9$.
This gives $B_s=K_b/K_a=0.5$, which meets the shape factor \num{1.5}.

The comparison can be seen in \figref{fig:nm_calibration}.
The overall hysteresis shape can be well captured, and the theoretical maximum moment ($M_{max}=3$) of the section is properly reflected.
Fine--tuning of hardening behaviour can be achieved by, for example, adopting a multiplicative formulation \citep{Chaboche1989}.
The saturation can be associated with the shape factor of the section, which is essentially the ratio between the plastic section modulus and the elastic section modulus, with both usually available in section property tables.
With a specific $N$-$M$ interaction surface, it is possible to adjust the response under various levels of axial loads.
\subsection{A Frame Example}
The performance of the proposed element is compared with that of conventional force-based fibre elements via a response history analysis of a three--storey steel frame structure.
The geometry and dimension of the frame are shown in \figref{fig:nm_frame_hinge}.
The floor mass is \SI{10}{kips\cdot{}s^2/ft} which is lumped on nodes according to tributary length/area and is applied to both horizontal and vertical DoFs.
A uniform damping ratio of \SI{5}{\percent}, which may be high for practical steel structures but is acceptable from the design perspective, using the bell-shaped damping model \citep{Lee2020a,Lee2020b,Lee2021,Lee2022} is defined to cover the response that falls in the frequency range from \SI{0.001}{\radian\per\second} to \SI{1000}{\radian\per\second}, which is sufficient for the target structure with only hardening response.
The material elastic modulus $E$ is set to $E=\SI{29000}{ksi}$ and the yield stress is set to $\sigma^y=\SI{50}{ksi}$.
Those values are used to compute section properties.
Two loads are applied to the structure: 1) vertical gravity load and 2) horizontal earthquake action.
The ground motion used is the normalised accelerogram of the 1940 El Centro earthquake with PGA equal $0.5g$.

The force-based fibre frame element \citep{Spacone1996} is used as the reference element.
The non-iterative algorithm proposed by \citet{Neuenhofer:Filippou:97:Evaluation} is implemented to optimise its computational performance.
Six integration points (sections) along the element chord are used.
For each section, ten integration points along the depth of the section are assigned, resulting in \num{60} material integration points per element.
For the material model, a bilinear hardening model with \SI{1}{\percent} linear isotropic hardening is used.

For the proposed beam element, \SI{1}{\percent} linear isotropic hardening is assigned, and kinematic hardening parameters are determined following the same aforementioned procedure. In specific, the hardening speed $K_a$ is set to unity $K_a=1$, which can be further calibrated, while $K_b/K_a=\dfrac{\text{Plastic Section Modulus}}{\text{Elastic Section Modulus}}-1$.
All section properties are taken from the design manual \citep{AISC2017}. \eqsref{eq:nm_surface_used} is chosen to be the $N$-$M$ interaction surface with $c=1.15$. The comparisons between \eqsref{eq:nm_surface_used} and the ones generated by section analysis is shown in \figref{fig:nm_surface_example}.

The roof drift history is given in \figref{fig:nm_frame_example}.
With the maximum drift being \SI{1.5}{\percent}, significant plasticity is developed as shown in \figref{fig:nm_frame_hinge}, but the drift history shows no significant difference.

In terms of computational cost, for this specific structure, in a parallel context, performing a full response history analysis with the proposed element takes around \SI{30}{\percent} less wall clock time. Note that the number is indicative and may vary depending on different configurations. Excluding the time spent on solving the global matrix, the cost of which is not relevant to element formulations, the proposed element takes around \SI{80}{\percent} less time in terms of sole state determination. Given that the solution of the global matrix typically accounts for a significant portion of the overall computational time, a reduction of \SI{30}{\percent} is regarded as decent.

To close this example, the hysteresis of frame members is presented in \figref{fig:hys}.
It must be pointed out that frame end hysteresis ($\theta_z$-$M_z$) does not reflect whether the target end yields. See the discussion around \eqsref{eq:redefine}.
The responses of components that are not shown (such as the axial forces of middle columns) are purely elastic.

In general, both elements exhibit a comparable hysteresis pattern. As all beams yield, a beam--sway mechanism is formed. However, due to the sensitivity to plastic deformation, the centre of hysteresis may shift by varying amounts.

For a similar bending moment response, the proposed element tends to exhibit larger axial deformation of beams, suggesting that further calibration is necessary for the $N$-$M$ interaction surface. This disparity may arise from multiple sources, including insufficient integration points utilised in fibre elements, inaccurate section discretisation, and varying hardening behaviour attributed to different formulations.

If desired, analysts may perform calibration or implement more sophisticated hardening rules, such as the one presented in the appendix. Nonetheless, as these topics fall beyond the scope of this work, they are not discussed further.

It is worth mentioning that there is no intent to fine-tune the element behaviour to match experimental data on a designation-by-designation basis. Instead, the above structure is presented as a practical example to demonstrate that, without a complex calibration procedure, the proposed element can generate results with a certain level of confidence due to the correct formulation of hardening rules.

Upon comparison with the reference force-based element, it becomes apparent that the results obtained from both elements are similar. There is no plausible reason to indicate that one element is superior to the other. In specific circumstances where the overall response is governed by the strain--stress response at the material level, fibre models tend to necessitate fewer calibration efforts for different members. With the current model, it is required to calibrate $N$-$M$ interaction surfaces on a member-by-member basis.
\section{Conclusions}
In this work, the generalised plasticity theory is revisited and applied to conventional beam elements.
By using a revised yield function, which is able to capture the yielding of either end(s), we formulate an efficient concentrated plasticity beam element that supports the $N$-$M$ interaction, which allows the definition of flexible sectional responses.
The proposed formulation adopts sectional deformations and resultants as the basic quantities and determines the plastic state using a return mapping algorithm.
As evidenced by the frame example, the proposed formulation demonstrates a fairly moderate reduction in computational costs when compared to fibre elements for the presented relatively simple structure.
This renders it an effective and performant tool for simulating lumped plastic hinges at beam ends with zero hinge length.

By properly accounting for the special constraint that the axial force is unique but shared between two end nodes, the hardening behaviour is corrected.
The proposed element shows desired isotropic/kinematic hardening response.
The hardening parameters are now tightly associated with physical implications, making the calibration more reliable.
As the presented hardening laws are suitable for steel beams/columns, practical examples are presented to showcase that the proposed element can be used in response history analysis of steel frames without a complex calibration procedure of model parameters.
It is imperative to note that the exact plastic rules delineated in this study are exclusively applicable to a particular case of steel members with a bilinear material stress--strain law, and are not suitable for sections made of alternative materials, such as reinforced concrete.
Nevertheless, with the proposed formulation, other plastic rules and interaction surfaces can be adopted to simulate other types of frames.
The formulation itself does not impose any restrictions in this regard.
It provides an alternative to conventional beam elements with flexibility for response history analysis.

As sectional quantities are employed as the basic quantities, the corresponding calibration of model parameters (hardening rules and $N$-$M$ interactions) should be carried out based on the corresponding experimental data of member tests or accurate sectional analysis results (in absence of experimental data), if the proposed model is to be used in predictive simulations.

The proposed beam elements (both 2D and 3D versions) are implemented in \texttt{suanPan} \citep{Chang2022}, and all numerical examples are analysed using the same application.
\appendix
\section{Component Based Kinematic Hardening Rule}
Consider a 3D frame element with singly symmetric section geometry, as in general the shape factor, which corresponds to the saturation level defined in the kinematic hardening rule, would have different values along different axes, it is desired to assign different hardening rules for different components.

A simple modification of \eqsref{eq:nm_kin} can be expressed as
\begin{gather}
\dot{\overline{\bbeta}}=\begin{bmatrix}
K_{b,P}&\cdot&\cdot\\
\cdot&K_{b,M_z}&\cdot\\
\cdot&\cdot&K_{b,M_y}
\end{bmatrix}\dot{\overline{\be}^p}-\begin{bmatrix}
K_{a,P}&\cdot&\cdot\\
\cdot&K_{a,M_z}&\cdot\\
\cdot&\cdot&K_{a,M_y}
\end{bmatrix}\norm{\dot{\overline{\be}^p}}\overline{\bbeta}.
\end{gather}
The $K_a$ and $K_b$ pairs can be independently calibrated for each component to match the corresponding section properties of strong/weak axes.

Furthermore, consider an idealised bilinear material model, under pure axial deformation, the axial force resistance follows the material model and generates a bilinear response. However, under pure bending deformation, the end moment develops plasticity gradually and generates a non-linear response, which asymptotically approaches the plastic moment after yielding. One can set $K_{a,P}=0$, $K_{a,M_z}\neq0$ and $K_{a,M_y}\neq0$ so that axial back resistance evolves linearly while bending resistance evolves non-linearly. A graphical illustration of 2D elements is given in \figref{fig:nm_component_kin}.
\section*{Data Availability Statement}
Some or all data, models, or code that support the findings of this study are available from the corresponding author upon reasonable request.
The corresponding model scripts are available online\footnote{\url{https://github.com/TLCFEM/nm-formulation}}.
\bibliography{BIB}
\end{document}